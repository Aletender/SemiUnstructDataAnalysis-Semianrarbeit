\section{Einleitung} 
Immer mehr Menschen konsumieren Nachrichten über digitale Medien. \footcite [Vgl.][] {IfDAllensbach2023}

\subsection{Zielsetzung}
Im Rahmen dieser Seminararbeit soll erarbeitet werden, wie der Entwurf und die Implementierung einer Datensammlungsumgebung optimiert werden kann, um die Herausforderungen beim Umgang mit semi-strukturierten und unstrukturierten Daten, zu überwinden.
Zur Beantwortung dieser Frage soll eine Strategie entwickelt und umgesetzt werden, welche in einer funktionalen Umgebung zum Sammeln von Nachrichten und Kommentaren resultiert und verschiedene Qualitätsmerkmale bedient. 

\subsection{Methodik}
Die Methodik dieser Arbeit gliedert sich auf drei aufeinander aufbauende Praktiken auf, mit dem Ziel, auf Basis aktueller Erkenntnisse einen Prototypen zu modellieren, zu entwickeln und zu evaluieren. 

\subsubsection{Literaturdurchsicht}
Zur Schaffung einer theoretischen Grundlage, beginnt diese Arbeit mit einer Literaturdurchsicht. Diese soll einen Überblick über aktuelle Technologien und bewährte Strategien gewähren. 
Als Ergebnis dieses Prozesses sollen verschiedene Aspekte herausgestellt worden sein, welche in der zu erarbeitenden Architektur zu beachten sind. Darunter fallen verschiedene Rahmenbedinungen durch die Datenquellen, wie auch spezielle Anforderungen durch die zu leistende Analyse der Daten.
Darüber hinaus soll aus der Literaturdurchsicht entspringen, welche Anforderungen sowohl an den Prozess der Datensammlung, sowie an den zu erstellenden Datenkorpus zu stellen sind. 

\subsubsection{Modellierung und Prototyping}
Auf Basis der gesammelten theoretischen Grundlagen soll eine Architektur inklusive der darin vorkommenden technischen Prozesse modelliert werden. 
Im Rahmen der architektonischen Entwicklungsarbeit werden insbesondere die zuvor identifizierten Rahmenbedingungen (engl. constraints), sowie zu messende Qualitätsmerkmale identifiziert und durch das entwickelte Konzept berücksichtigt. 
Zudem ist in der Architektur zu Identifizieren, welcher Anteil minimal zu leisten ist, um ein funktionsfähiges System bereitzustellen (engl. Minimum Viable Product, MVP).
Dieser Anteil der Architektur soll in der Form eines Prototypen implementiert werden. Dadurch soll evaluiert werden, ob die konzipierte Architektur funktional ist und den beabsichtigten Datenkorpus bereitstellen kann.

\subsubsection{Technische Analyse }
Die letzte Methodik folgt der Absicht, den Prototypen zu evaluieren und den technischen Durchstich abzuschließen. Dazu wird eine, in der Literaturdurchsicht, bestimmt Analyse auf den gebildeten Datenkorpus angewandt, um dessen Eignung festzustellen. 
Darüber hinaus werden die gewählten Qualitätsmerkmale durch festgestellte Verfahren geprüft. 